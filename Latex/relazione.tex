\documentclass[12pt]{article} % Prepara un documento con un font grande
\usepackage[italian]{babel} % Adatta LaTeX alle convenzioni tipografiche italiane,
							% e ridefinisce alcuni titoli in italiano, come "Capitolo" al posto di "Chapter",
							% se il documento è in italiano

\usepackage[utf8]{inputenc} % Consente l'uso caratteri accentati italiani
\usepackage{graphicx}		% Per le immagini


\title{Titolo del documento}
\author{Francesco Forcher\\ % la doppia barra inversa forza a capo
Facoltà di Fisica\\
Università di Padova\\
Matricola: \texttt{1073458}\\
\texttt{mailto:francesco.forcher@studenti.unipd.it}\\
\and
Francesca Damiani\\ % Secondo autore indicato con \and  1070620
Facoltà di Fisica\\
Università di Padova\\
Matricola: \texttt{1071072}\\
\texttt{mailto:francesca.damiani@studenti.unipd.it}\\
\and
Andrea Piccinin\\ % Secondo autore indicato con \and  1070620
Facoltà di Fisica\\
Università di Padova\\
Matricola: \texttt{1070620}\\
\texttt{mailto:andrea.piccinin1@studenti.unipd.it}\\
}

\date{\today}


\pagestyle{headings}
\DeclareGraphicsExtensions{.svg, .pdf, .png, .jpg} % Se due immagini hanno lo stesso nome sceglile secondo l'ordine di filetype qui

\graphicspath{ {./img/} }%%%%%%%%%%%%%%%%%%%%%%%%%%%%%%%%EDITARE PATH!!%%%%%%%%%%%%%%%%%%%%%%%%%%%%%%%%%%%%%%%%% % Path delle immagini 















% PROVA MODIFICA DA BROWSER!!!!
%////////////////////////////////////////////////////////////////////////////////////////////////////////////////////////////
%////////////////////////////////////////////////////////////////////////////////////////////////////////////////////////////
% Fine dei dati iniziali per il latex: il documento finale inizierà da qui
\begin{document}


\maketitle % Produce il titolo a partire dai comandi \title, \author e \date
\tableofcontents % Prepara l'indice generale


\begin{abstract} % Questo è l'inizio dell'ambiente "abstract" ("Sommario")
	% L'ambiente abstract è fatto per contenere un riassunto del contenuto.
	RIASSUNTO
\end{abstract}


\section{Titolo prima parte}
	Testo introduttivo prima parte.
	È possibile scrivere il testo dell'articolo normalmente, ed \emph{enfatizzare} alcune parti del discorso. %
	Una riga vuota nel testo indica la fine di un paragrafo.
 	
	Paragrafo 2


\subsection{Paragrafo della prima parte}
	Testo generico

	Paragrafo 1
	\linebreak[4] % Inserisci uno spazio bianco tra i paragrafi, il numero da 0 a 4 è la priorità (quanto è importante che ci sia questo spazio)
	Paragrafo 2 



\paragraph{Esempio inserimento codice}
	Esempio di codice inserito nel testo
	%Verbatim non interpreta l'imput lasciando il testo com'è: ideale per inserire codice
	\begin{verbatim}
	#include <iostream>

	int main() {
		using namespace std;\\\\\\\\\\\\\\\\\\\\\\\\\\\\\\\\\\\\\\\\\\\\\\\\\\\\\\\\\\\\\\\\\\\\\\\\\\\\\\\\\\\\\\\\\\\\\\\\\\\\\
		cout << "Hello world"; \\Stampa hello world
		return 0;
	}
	\end{verbatim}

	Testo paragrafo: formule %Esempio formula
	esempio: \((x+2)^n\)


\paragraph{Esempio inserimento tabella dati}
Test tabella
\begin{table}[!ht]
	\small
	\centering
	\caption{Esempio Tabella dei dati}
	\begin{tabular}{|l|r|r|r|r|r|r|}
		\hline
		\multicolumn{1}{|c|}{\textbf{Intervalli}} & \multicolumn{1}{c|}{40 – 60} & \multicolumn{1}{c|}{50 – 70} & \multicolumn{1}{c|}{60 – 80} & \multicolumn{1}{c|}{70 – 90} & \multicolumn{1}{c|}{80 – 100} & \multicolumn{1}{c|}{90 – 110} \\ \hline
		 & \multicolumn{ 6}{c|}{} \\ \hline
		\multicolumn{ 1}{|c|}{\textbf{Dati}} & 1.664 & 1.660 & 1.728 & 1.705 & 1.724 & 1.662 \\ \cline{ 2- 7}
		\multicolumn{ 1}{|l|}{} & 1.642 & 1.666 & 1.758 & 1.697 & 1.774 & 1.681 \\ \cline{ 2- 7}
		\multicolumn{ 1}{|l|}{} & 1.668 & 1.680 & 1.736 & 1.723 & 1.772 & 1.702 \\ \cline{ 2- 7}
		\multicolumn{ 1}{|l|}{} & 1.630 & 1.668 & 1.706 & 1.726 & 1.714 & 1.741 \\ \cline{ 2- 7}
		\multicolumn{ 1}{|l|}{} & 1.637 & 1.666 & 1.726 & 1.760 & 1.752 & 1.710 \\ \hline
		\multicolumn{1}{|c|}{\textbf{Analisi}} & \multicolumn{1}{l|}{} & \multicolumn{1}{l|}{} & \multicolumn{1}{l|}{} & \multicolumn{1}{l|}{} & \multicolumn{1}{l|}{} & \multicolumn{1}{l|}{} \\ \hline
		Media & 1.6482 & 1.668 & 1.7308 & 1.7222 & 1.7472 & 1.6992 \\ \hline
		Varianza del campione & 0.00022736 & \multicolumn{1}{l|}{4.32e-05} & 0.00028256 & 0.00047496 & 0.00059936 & 0.00071736 \\ \hline
		Deviazione standard campione & 0.0150785 & 0.00657267 & 0.0168095 & 0.0217936 & 0.0244818 & 0.0267836 \\ \hline
		Varianza della popolazione & 0.0002842 & \multicolumn{1}{l|}{        5.4e-05} & 0.0003532 & 0.0005937 & 0.0007492 & 0.0008967 \\ \hline
		Deviazione standard popolazione & 0.0168582 & 0.00734847 & 0.0187936 & 0.024366 & 0.0273715 & 0.0299449 \\ \hline
		Errore della media & 0.00753923 & 0.00328634 & 0.00840476 & 0.0108968 & 0.0122409 & 0.0133918 \\ \hline
		Massimo & 1.668 & 1.68 & 1.758 & 1.76 & 1.774 & 1.741 \\ \hline
		Minimo & 1.63 & 1.66 & 1.706 & 1.697 & 1.714 & 1.662 \\ \hline
	\end{tabular}
	\label{}
\end{table}

% Esempio di inclusione di un immagine ./img/spazio1.png etichettata fig:spazio, al centro, larga 0.8 per larghezza del testo, con testo sotto Spazio!
%\subsection{Esempio immagini}
%\begin{figure}[!h]//Guardate Latex/Primo test per mettere l'immagine nella sottosessione giusta
%    \centering
%    \includegraphics[width=0.8\textwidth]{spazio1}
%    \caption{Spazio!}
%    \label{fig:spazio1}
%\end{figure}

\end{document}
