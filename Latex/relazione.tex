\documentclass[12pt]{article} % Prepara un documento con un font grande
\usepackage[italian]{babel} % Adatta LaTeX alle convenzioni tipografiche italiane,
							% e ridefinisce alcuni titoli in italiano, come "Capitolo" al posto di "Chapter",
							% se il documento è in italiano

\usepackage[utf8]{inputenc} % Consente l'uso caratteri accentati italiani
\usepackage{graphicx}		% Per le immagini


\title{Titolo del documento}
\author{Francesco Forcher\\ % la doppia barra inversa forza a capo
Facoltà di Fisica\\
Università di Padova\\
Matricola: \texttt{1073458}\\
\texttt{mailto:francesco.forcher@studenti.unipd.it}
\and
Pinco pallino\\ % Secondo autore indicato con \and
Facoltà di Fisica\\
Università di Padova\\
Matricola: \texttt{1234567890}\\
\texttt{mailto:pinco.pallino@ciao.it}\\
}
\date{\today}


\pagestyle{headings}
\DeclareGraphicsExtensions{.svg, .pdf, .png, .jpg} % Se due immagini hanno lo stesso nome sceglile secondo l'ordine di filetype qui

\graphicspath{ {./img/} }%%%%%%%%%%%%%%%%%%%%%%%%%%%%%%%%EDITARE PATH!!%%%%%%%%%%%%%%%%%%%%%%%%%%%%%%%%%%%%%%%%% % Path delle immagini 


%////////////////////////////////////////////////////////////////////////////////////////////////////////////////////////////
%////////////////////////////////////////////////////////////////////////////////////////////////////////////////////////////
% Fine dei dati iniziali per il latex: il documento finale inizierà da qui
\begin{document}


\maketitle % Produce il titolo a partire dai comandi \title, \author e \date
\tableofcontents % Prepara l'indice generale


\begin{abstract} % Questo è l'inizio dell'ambiente "abstract" ("Sommario")
	% L'ambiente abstract è fatto per contenere un riassunto del contenuto.
	Breve dimostrazione dell'uso di \LaTeX.
\end{abstract}


\section{Titolo prima parte (1.)}
	Testo introduttivo prima parte.
	È possibile scrivere il testo dell'articolo normalmente, ed \emph{enfatizzare} alcune parti del discorso. %
	Una riga vuota nel testo indica la fine di un paragrafo.
 
	Paragrafo 2


\subsection{Paragrafo della prima parte (1.1)}
	Testo generico

	Paragrafo 1
	\linebreak[4] % Inserisci uno spazio bianco tra i paragrafi, il numero da 0 a 4 è la priorità (quanto è importante che ci sia questo spazio)
	Paragrafo 2



\paragraph{Esempio inserimento codice}
	Esempio di codice inserito nel testo
	%Verbatim non interpreta l'imput lasciando il testo com'è: ideale per inserire codice
\begin{verbatim}
#include <iostream>

int main() {
	using namespace std; \\\\\\\\\\\\\\\\\\\\\\\\\\\\\\\\\\\\\\\\\\\\\\\\\\\\\\\\\\\\\\\\\\\\\\\\\\\\\\\\\\\\\\\\\\\\\\\\\\\\\
	cout << "Hello world"; \\Stampa hello world
	return 0;
}
\end{verbatim}

	Testo paragrafo: formule %Esempio formula
	esempio: \((x+2)^n\)

% Esempio di inclusione di un immagine ./img/spazio1.png etichettata fig:spazio, al centro, larga 0.8 per larghezza del testo, con testo sotto Spazio!
%\subsection{Esempio immagini}
%\begin{figure}[p]
%    \centering
%    \includegraphics[width=0.8\textwidth]{spazio1}
%    \caption{Spazio!}
%    \label{fig:spazio1}
%\end{figure}

\end{document}
