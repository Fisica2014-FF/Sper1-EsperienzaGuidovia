\documentclass[12pt]{article} % Prepara un documento con un font grande
\usepackage[italian]{babel} % Adatta LaTeX alle convenzioni tipografiche italiane,
							% e ridefinisce alcuni titoli in italiano, come "Capitolo" al posto di "Chapter",
							% se il documento è in italiano

\usepackage[utf8]{inputenc} % Consente l'uso caratteri accentati italiani
\usepackage{graphicx}		% Per le immagini
\usepackage[top=2in, bottom=1.5in, left=0.5in, right=0.5in]{geometry}
\usepackage{float}

\title {Relazione di Laboratorio - Guidovia}
\author{Francesco Forcher\\
Facoltà di Fisica\\
Università di Padova\\
Matricola: \texttt{1073458}\\
\texttt{mailto:francesco.forcher@studenti.unipd.it}\\
\and
Francesca Damiani\\ 
Facoltà di Fisica\\
Università di Padova\\
Matricola: \texttt{1071072}\\
\texttt{mailto:francesca.damiani@studenti.unipd.it}\\
\and
Andrea Piccinin\\ 
Facoltà di Fisica\\
Università di Padova\\
Matricola: \texttt{1070620}\\
\texttt{mailto:andrea.piccinin1@studenti.unipd.it}\\
}

\date{\today}


\pagestyle{headings}
\DeclareGraphicsExtensions{.svg, .pdf, .png, .jpg} % Se due immagini hanno lo stesso nome sceglile secondo l'ordine di filetype qui

\graphicspath{ {../Gnuplot/immagini/} }%%%%%%%%%%%%%%%%%%%%%%%%%%%%%%%%EDITARE PATH!!%%%%%%%%%%%%%%%%%%%%%%%%%%%%%%%%%%%%%%%%% % Path delle immagini 















% PROVA MODIFICA DA BROWSER!!!!
%////////////////////////////////////////////////////////////////////////////////////////////////////////////////////////////
%////////////////////////////////////////////////////////////////////////////////////////////////////////////////////////////
% Fine dei dati iniziali per il latex: il documento finale inizierà da qui
\begin{document}


\maketitle % Produce il titolo a partire dai comandi \title, \author e \date
\tableofcontents % Prepara l'indice generale



        

\section{Obiettivi}
	Stimare il valore dell'accelerazione di gravità \textbf{g} attraverso la misurazione dell'accelerazione della slitta su un piano 
	inclinato, tenendo conto, nella seconda parte dell'esperienza, anche del contributo dell'attrito dell'aria.

\section{Descrizione dell'apparato strumentale}
	Slitta in plexiglass che, inizialmente bloccata da un elettromagnete, scorre lungo una guidovia di acciaio, il cui attrito con la 		slitta è rimosso da un cuscino d'aria. 
	Traguardi mobili a sensori infrarossi, collegati ad un cronometro di sensibilità \(10^{-3}\) s.
	Nella seconda parte dell'esperienza , essendo la guida posta orizzontalmente, la slitta è stata fatta muovere attraverso un 		impulso elettrico dato dall'elettromagnete, che ne determina la velocità iniziale. 
	

\section{Metodologia di misura}
	Nella prima parte dell'esperienza, l'elettromagnete che trattiene la slitta viene disattivato tramite un pulsante. La slitta inizia 		così a scorrere lungo la guidovia su cui sono posizionati i traguardi, che inviano al cronometro il tempo di percorrenza 		dell'intervallo stabilito. Le misure vengono prese partendo da 40 cm dall'elettromagnete, aumentando l'ampiezza dell'intervallo da 10 		cm a 70 cm. Le misure vengono poi ripetute applicando un disco di ottone sulla slitta in modo da modificarne il peso.
	Nella seconda parte dell'esperienza, essendo la guidovia orizzontale, si fa muovere la slitta attraverso un impulso elettrico, che ne 		determina la velocità iniziale. In questo caso le misure vengono prese su intervalli di 20 cm, sempre partendo da 40 cm di distanza 		dall'elettromagnete, in modo da stimare una riduzione della velocità dovuta alla forza di attrito. La velocità iniziale viene poi 		modificata interponendo tra la slitta e l'elettromagnete uno spessore in alluminio.

\section{Presentazione dati sperimentali}	
	Riportiamo in seguito le misure tabulate, con relative statistiche.
	\subsection {Inclinazione 15', senza peso}
		La quarta misura nell'intervallo 40-60 cm ha un valore non compatibile con gli altri. Potrebbe esserci stato un errore nella 			trascrizione dei dati durante la misurazione. 
	\subsection {Inclinazione 30', senza peso}
	\subsection {Inclinazione 45', senza peso}
	\subsection {Inclinazione 45', con peso}
	\subsection {Inclinazione 0', senza peso, senza spessore}
	\subsection {Inclinazione 0', senza peso, con spessore}
	\subsection {Inclinazione 0', con peso, senza spessore}
	\subsection {Inclinazione 0', con peso, con spessore}
\section{Discussione dati sperimentali}
	In seguito vengono allegate due tabelle rappresentanti un elaborazione delle misure prese durante le esperienze:
	\subsection {tabella Coefficienti_Angolari}
	\subsection {tabella Stime_Accelerazioni_di_Gravita.ods DA FINIRE}
Nella prima tabella vengono esposti tutti i valori dei coefficienti angolari, con il relativo errore, estrapolati (tramite il programma grafico: Gnuplot) dalle rette illustrate nelle precedenti rappresentazioni;
Nella seconda tabella invece vengono mostrate le stime dell'accelerazione di gravità (nella sezione riguardante la prima parte dell'esperienza) e i loro fattori di correzione (nella sezione riguardante la seconda parte dell'esperienza).  
Le stime dell'accelerazione di gravità, \textbf{g}, ricavate dalla prima serie di esperimenti, ovvero lavorando con la guidovia a varie inclinazioni (15', 30', 45' con e senza peso) forniscono una stima dell'accelerazione di gravità media, \textbf{g}\ped{0}, pari a (9.21 $\pm$ 0.08) m$\cdot$s\ap{-2}. Già a prima vista si può notare che esso differisce di molto rispetto al valore atteso a padova, \textbf{g}\ped{p} = (9.806 $\pm$ 0.001) m$\cdot$s\ap{-2}. Matematicamente quest'osservazione è confermata dal valore della compatibilità tra le due misure, ovvero 7.45; Queste sono chiaramente incompatibili. Probabilmente ciò è dovuto a causa degli errori sperimentali commessi durante l'esecuzione dell'esperimento dagli operatori.
Il valore appena calolato \textbf{g}\ped{0} è comunque soggetto ad un errore dovuto alle forze di attrito agenti sull'apparato sperimentale. Questo errore è stato corretto di un fattore $\Delta$\textbf{g}, calcolato nella seconda serie di esperimenti, in cui la guidovia è stata tenuta con inclinazione nulla. La media dei quattro valori di correzione (no peso, no spessore; no peso, spessore; peso, no spessore; peso, spessore) è pari a: (\textbf{????} $\pm$ \textbf{????}) m$\cdot$s\ap{-2};
per cui la nostra stima dell'accelerazione di gravità, corretta dalle forze di attrito, è pari a \textbf{g} = \textbf{g}\ped{0} + $\Delta$\textbf{g} = (9.42 $\pm$ 0.09) m$\cdot$s\ap{-2}.
La compatibilità di questo valore rispetto al valore atteso a padova è pari a: 4.29, ovvero è ancora non compatibile, nonostante, com'è lecito aspettarsi, sia comunque una miglior stima rispetto al valore non corretto dalle forze di attrito. 
\section{Conclusioni}
	Per verificare quale sia il metodo migliore per stimare l'accelerazione di gravità si allega questa tabella in cui sono riassunti tutti i valori di \textbf{g}, corretti dall'errore dovuto all'attrito, correlati alla relativa compatibilità rispetto al valore atteso a padova, \textbf{g}\ped{p}.
	\subsection {$Tabella Stime_Gravita_corrette.ods DA FINIRE$}
Da questa si evince che il miglior metodo per calcolare l'accelerazione di gravità è: .........(\textbf{TOMORROW MORNING WORK})



% Esempio di inclusione di un immagine ./img/spazio1.png etichettata fig:spazio, al centro, larga 0.8 per larghezza del testo, con testo sotto Spazio!
\section {Grafici}
	Riportiamo in seguito i grafici delle velocità medie e le rette interpolanti.
	
\subsection{Esempio immagini}
\begin{figure}[H]
    \centering
    \includegraphics[width=0.8\textwidth]{spazio1}
    \caption{Spazio!}
    \label{fig:spazio1}
\end{figure}

\section{Codice}
	Riportiamo in seguito il programma utilizzato per l'elaborazione dei dati
	%Verbatim non interpreta l'imput lasciando il testo com'è: ideale per inserire codice
	\begin{verbatim}
	#include <iostream>

	int main() {
		using namespace std;\\\\\\\\\\\\\\\\\\\\\\\\\\\\\\\\\\\\\\\\\\\\\\\\\\\\\\\\\\\\\\\\\\\\\\\\\\\\\\\\\\\\\\\\\\\\\\\\\\\\\
		cout << "Hello world"; \\Stampa hello world
		return 0;
	}
	\end{verbatim}
\end{document}
